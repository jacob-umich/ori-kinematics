\newpage
%\Large{\textbf{Abstract}}
\section*{Abstract}
Kinematic simulations are often the first step in analyzing the motion of a body or set of bodies. In this project, a library was developed that can produce a kinematically admissible motion given a target motion and rigid bodies in the system. This library was created to simulate the kinematics of thick origami. The library was developed according to a waterfall development pattern whereby multiple steps were taken prior to writing code to ensure efficient code architecture. This included creating a formal problem definition consisting of high-level and functional requirements and class and state diagrams.

This project also employed common software development practices. In particular, all code and documentation was handled within GitHub. Here, issues and tasks could be delegated to individual group members to complete the high-level design, low-level design, construction, and testing portions of the code. Further, CMake was used for linking and makefile generation. Object oriented design and version control through Git were utilized throughout. External performant libraries were used for efficient matrix computations. 

The developed library is able to take a JSON file input corresponding to some set of rigid bodies and the desired target velocity and output a visualization showing the folding motion of the bodies. Individual portions of the code were tested and verified before running full example scenarios. The results for two different folding examples are shown with some limitations of the simulations discussed. This library was shown to be able to successfully simulate kinematically admissible motion of input rigid bodies. Additionally, designers of the library gained valuable experience in kinematic simulations and proper software engineering practices. Future work would seek to improve the simulation by simplifying the input file generation, adding more realistic physics, and reducing error through better iteration schemes. 